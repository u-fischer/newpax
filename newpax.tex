\makeatletter
\def\UlrikeFischer@package@version{0.1}
\def\UlrikeFischer@package@date{2020-03-30}
\makeatother
\RequirePackage[patches]{pdfresources}
\DeclareDocumentMetaData{pdfversion=1.7,lang=en-UK}
\documentclass[DIV=12,parskip=half-,bibliography=totoc]{scrartcl}
\usepackage{scrlayer-scrpage}
\usepackage{fontspec}
\setmainfont{Heuristica}
\usepackage{unicode-math}
\usepackage[english]{babel}
\usepackage[autostyle]{csquotes}
\usepackage{microtype}

\usepackage{listings}
\lstset{basicstyle=\ttfamily, columns=fullflexible,language=[LaTeX]TeX,
        escapechar=*,
        commentstyle=\color{green!50!black}\bfseries}
\usepackage[customdriver=hgeneric-experimental,
             pdfdisplaydoctitle=true,pdfusetitle,hyperfootnotes=false,
            ]{hyperref}
\usepackage{newpax}

\title{The \pkg{newpax} package, v\csname UlrikeFischer@package@version\endcsname}
\subtitle{Reinserting annotations from included pdf file}
\date{\csname UlrikeFischer@package@date\endcsname}
\author{Ulrike Fischer\thanks{fischer@troubleshooting-tex.de}}

\begin{document}
\maketitle

\section{Introduction}

When including pdf-files in a document -- may it be with \verb+\includegraphics+ or with \verb+\includepdf+ -- clickable links and other annotations in this documents are lost.

The  \pkg{newpax} package offers some tools to reinsert these annotations. It is based in large parts
on the \pkg{pax} package from Heiko Oberdiek. 

\subsection{Importing annotations}

Clickable links in a pdf are one example of an annotation. Annotations are areas on a page which are associated with an action. A typical annotation object could look this this:

\begin{lstlisting}
15 0 obj
<<
/Type /Annot
/Border[0 0 1]/BS<</S/U/W 1>>/H/I/C[0 1 1]
/Rect [147.716 654.025 301.887 665.15]
/Subtype/Link/A<</Type/Action/S/URI/URI(https://www.latex-project.org)>>
>>
endobj
\end{lstlisting} 

The \texttt{/Rect} value describes the rectangle of the annotation. It is important to understand that an annotation is not connected to some content but to a page. And the coordinates are absolute page coordinates. 

 
To \enquote{reactivate} the annotations of an included pdf one has to
\begin{itemize}
\item retrieve and store the annotations of the included pdf. For links to external links to requires to find only one object. But e.g. internal links point to a destination object and these must be found too.
\item recalculate the rectangle coordinates to fit to the coordinate system of the target page: as the included pdf can be scaled, rotated and even clipped this is not an easy task. Destinations have rectangles too that must be recalculated.
\item  Reinsert the annotation and related objects. This has to take into account the a pdf is perhaps not included completly, a link shouldn't point to a missing page or a clipped annotation. It also has to take into account that a pdf is perhaps inserted more than once or in steps. 
\end{itemize}

\section{Importing annotations with the \pkg{pax} package}

The \pkg{pax} package from Heiko Oberdiek consists of a script \texttt{PDFAnnotExtractor}
which extract the data of all annotations from a pdf and stores them in a text file with the extension \texttt{pax}, a style \texttt{pax.sty} which patches the \verb+\includegraphics+ command to load the \texttt{pax}, process it to calculate the rectangles and reinsert destinations and annotations.

The package works actually very well. Problems are
\begin{itemize}
\item \texttt{PDFAnnotExtractor} requires an external, old version of the java library of PDFbox.
\item The style works only with pdflatex and partly with lualatex (with the luatex85) and not at all with xelatex. 
\end{itemize}

\section{Importing annotations with the \pkg{newpag} package}



\end{document}
